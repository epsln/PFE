

\subsection {Contexte}

Ce projet est proposé par la Prefecture de la Mayenne dans un contexte de dématérialisation des documents administratifs.
En effet dans un effort de numérisation de leur base documentaire, il devient complexe de gérer la grande quantité/diversité de documents informatisés. 

\par
La Prefecture de Mayenne envisage donc un projet de numérisation et classification automatisée avec l'aide des nouvelles technologies.

La mise en oeuvre de ce projet permettra de réduire la masse documentaire de l'État.


\subsection {Analyse du problème}
La Prefecture dispose d'un grand nombre de documents manuscrits qui nécessitent d'être numérisés et classés.
Ce travail laborieux est actuellement réalisé par des humains, ce qui rend cette tâche couteuse en temps et en ressources humaines.


La Prefecture envisage donc la mise en place d'un système automatisant toute la chaine de travail, de la numérisation à la classification.
Afin de faciliter l'accès aux documents numérisés, la Prefecture désire aussi mettre en place un moteur de recherche afin de réduire le temps de recherche documentaire.


\subsection {Critères de satisfaction}
TODO


