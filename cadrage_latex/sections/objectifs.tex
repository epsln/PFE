

\subsection {Objectifs du projet}

Les objectifs du projet on été définis avec le commanditaire lors de la première réunion.
Pour que le projet soit fonctionnel, il doit :
\begin {itemize}
\item être open source
\item analyser un document numérisé
\item permettre la recherche d'un document dans le base de données
\newline
\end {itemize}
Ces objectifs ont été précisés ci dessous afin d'en tirer des étapes plus précises selon notre plan IVVQ. \\ 

Ci dessous, les objectifs précis ont été redéfinis :
\begin {itemize}
\item être open source et réalisé avec des technologies open source
\item récupérer le contenu d’un document numérisé
\item analyser le contenu sémantique d’un document
\item représenter un document sous une forme vectorielles
\item classer les documents selon une taxonomie établie
\item ajouter à chaque document numérisé des métadonnées en format CEDAF
\item trouver les versions mineures d’un document déjà stocké
\item permettre la recherche d’un document par taxonomie
\item permettre la recherche d’un document par similarité apparente
\item permettre la recherche d’un document par similarité du contenu
\item permettre la recherche des documents associés à une personne
\item permettre la recherche des documents associés à un document
\end {itemize}

Ces objectifs ont été définis lors de la rédaction du plan IVVQ et ont été validés avec le commanditaire.
Ils seront à la base des vérification que nous effectuions avec le commanditaire à la fin de la l'année.
Ces objectifs sont aussi sujet à évolution car nous travaillons avec les méthodes agiles avec notre commanditaire.


\subsection {Objectifs du groupe}
Notre groupe à plusieurs objectifs pour ce projet.
Tout d'abord nous voulons aborder ce projet comme une entreprise le ferait.
C'est pourquoi nous allons mettre en place un plan IVVQ (Intégration Vérification Validation Qualification) à présenter et faire valider par notre commanditaire.
Ce plan sera la référence que nous utiliserons pour vérifier et valider notre réalisation avec le commanditaire à la fin de l'année, et nous sera aussi utile pour le planning prévisionnel et la marche à suivre lors de la production.
Nous avons également pour objectif d'utiliser les méthodes agiles dans un contexte réel.
Jusqu'ici l'application de ces méthodes à été pour nous un exercice qu'il était facile de tordre pour respecter la consigne.
Dans ce projet, il à été défini avec le commanditaire que nous travaillerons avec lui avec la méthode agile.


D'un point de vue technique, nous voulons profiler nos compétences en machine learning, surtout appliqué au domaine de l'analyse naturelle de texte (NLP).
Nous pourrons aussi expérimenter avec les technologies de moteur de recherche, la deuxième partie importante du projet.


A la fin du projet, nous voulons avoir produit un POC fonctionnel qui sera satisfaisante pour le commanditaire et qui lui donnera envie de revenir travailler avec l'ESIEA l'année prochaine.





