\subsection {Problématique et enjeux}
%Quel est le probleme
%Qui a le probleme
%Comment le probleme se présente t'il
%En quoi est-ce un probleme
%Quels sont les enjeux ?
La préfecture de la Mayenne doit, par obligation légale, conserver la totalité des documents administratifs produit et traités. La durée de conservation dépend du type de document; cela peut aller a quelques années pour des documents mineurs, à une durée indéfinie. En faisant ainsi, l'administration garde une trace de ses actions et est censé permettre une certaine transparence. 

Cependant, la production documentaire de l'administration est tout a fait considérable, et cela d'autant plus avec l'arrivé de l'informatique. Chaque jour des dizaines de nouveaux documents, arrêtés, formulaires... sont crées et doivent être gardés. Il n'y a pas, ou très peu de règles permettant une uniformisation de l'organisation de ces documents. Bien qu'il existe une taxonomie officielle, qui permet de définir très précisement le type d'un document et son contenu, celle ci est bien trop lourde et complexe pour qu'elle soit utilisée naturellement. Chaque employé administratif utilise donc sa propre méthode organisationnelle, et range, ou tente de ranger les gigabytes de documents qu'il doit traiter. Ce manque d'uniformité et de rationnalité dans l'organisation contribue a un ralentissement du processus administratif, qui nécessite par sa nature une grande quantité de document très précis qui ne peuvent pas être trouvé facilement dans la masse. Pour la préfecture de la Mayenne, on aurait affaire a environ 80 téraoctets de documents divers et variés. 

Ce problème est ressenti par la préfecture de la Mayenne, mais pas que; on le retrouve chez toute organisation d'envergure qui produit une grande quantité de document. 

Il est important de préciser que si le document en particulier répond a une organisation précise, et est bien défini \textit{en lui même}, il ne l'est pas dans \textit{l'ensemble}. En somme, si nous pouvons définir très précisement quel type de document il s'agit, il est difficile de le retrouver dans la masse de document produit. Il n'y quasiment jamais de métadonnées attachées a celui-ci, et il peut être caché dans une forêt de répertoire a la nomenclature différente pour chaque employé administratif. De plus, les documents administratifs possèdent des versions mineures et majeures. Les versions mineures sont des versions de travail, qui peuvent contenir des fautes et sont destinés a être paufiner pour devenir des documents majeurs propre a destination du grand public. Il arrive parfois que les versions mineures de ces documents soient conservés, ce qui complique encore plus la tâche.

Il s'agit là de rationaliser l'organisation documentaire de la préfecture, et de permettre de rendre plus efficace le processus administratif qui est considérablement ralenti par la recherche constante du bon document. Ce ralentissement a bien évidement un cout monétaire important, et est très frustrant pour les employés administratifs comme pour les citoyens ayant affaire a l'administration, que ce soit en Mayenne ou en France de façon générale. 

