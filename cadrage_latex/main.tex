\documentclass[french, 11pt, a4paper]{article}
%lexique des termes difficiles

\usepackage[utf8]{inputenc}
\usepackage[french]{babel}
\usepackage[T1]{fontenc}
\usepackage{hyperref}
\usepackage{geometry}
\usepackage{fancyhdr}
\usepackage{etoolbox} 
\usepackage{graphicx}
\usepackage{import} 
\usepackage{array}
\usepackage{afterpage}

\geometry{a4paper, margin=20mm}
\hyphenation {con-sti-tu-tio-nal}

%document begin
\pagestyle {plain}
\begin{document}

%Document main
\thispagestyle {plain}

\import {sections/} {pagegarde}
\newpage

\tableofcontents
\newpage



\part {Presentation}
\section {Contexte et finalité}
%Quelle est l 'entreprise
%pour qui est fait le projet
%Quel est le périmètre du projet
\import {sections/} {context}

\newpage
\section {Interlocuteurs}
\import {sections/} {interlocuteurs}



\newpage
\part {Problématisation}

\section {Problématique}
%Quel est le pb ? qui ? quel contexte ?
%En quoi est ce un pb ? critères de satis
\import {sections/} {problematique}

\section {Analyse du scénario}
\import {sections/} {analyseScenario}

\section {Objectifs}
%SMART (avec le commanditaire)
\import {sections/} {objectifs}



\newpage
\part {Cadrage}

\section {Livrables}
%liste des livrables
\import {sections/} {livrables}

\section {Démarche}
%Approche adoptée pour mener à bien le projet
%justification de l'approche
%Principales étapes
\import {sections/} {methode}

\section {Moyens}
%budget estimé
\import {sections/} {moyens}

\section {Organisation de l'équipe}
\import {sections/} {orgaequipe}

\newpage
\section {Planning}
%Gantt
Le planning se présente sous la forme d'un diagramme de Gantt, montré ci dessous, qui nous permet de comparer le travail effectué et le travail prévu.
Ce Gantt est disponible en format global et en format détaillé sur le SharePoint CapProjet.
Un planning prévisionnel des tâches est également crée chaque semaines de travail.
Il est accessible dans le même document que le Gantt, dans les onglets.

\begin{figure}[h!]
	\includegraphics[width=\linewidth]{images/gantt.png}
	\caption{Gantt prévisionnel et post mortem}
	\label{fig:MC}
\end{figure}		

\section {Gouvernance}
%réunions, échanges, lieux de travail
\import {sections/} {fonctionnement}

\newpage
\end {document}

