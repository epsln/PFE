
\section{Projet XL}
\subsection{Organisation}

Suite à l'évaluation des charges de travails que nous avions besoin pour terminer le projet, nous avions fait une demande de projet XL qui consiste à remplacer nos heures de mineurs managériales par la réalisation de notre PFE\@. 

Ce projet XL aura duré trois semaines 06/01/2020 - 24/01/2020.
Dans un premier temps, nous avons jugé qu'il serait plus intéressant de commencer par revoir notre organisation.
Il faut que chaque jour, nous avancions sur une tâche et que nous voyons l'avancement du projet.
C’est la raison pour laquelle nous avons décidé d’appliquer au mieux la méthode agile Scrum. 

Après avoir listé l’ensemble des tâches, nous les avons divisés en deux sprints:
\begin{itemize}
\item Sprint 1:
Version 1 PoC.
Semaine 06/01/2020 - 10/01/2020
\item Sprint 2:
Version 2 PoC et DER\@.
Semaine 13/01/2020 - 05/02/2020
\end{itemize}

Le sprint 2 a une durée plus élevée que le premier sprint dû à la différente charge de travail.
Il correspond en partie aux tâches non techniques (DER, Grand Oral et Soutenance).

Comme outil, nous avons utilisé Trello qui est un outil de gestion de projet. 

\subsubsection{Trello}
Trello est un outil de gestion de projet en ligne, lancé en septembre 2011 et inspiré par la méthode Kanban de Toyota.
Il repose sur une organisation des projets en planches listant des cartes, chacune représentant des tâches.

Durant le projet XL, nous avons utilisé cet outil pour sa visualisation et son efficacité avec les méthodes agiles. 

Pour mieux s’organiser, nous avons divisé les tâches en six parties ayant chacun un code couleur:
\begin{itemize}
\item Métadonnées: cyan
\item Taxonomie: bleu foncé
\item Moteur de recherche (JSON et interface): vert
\item DER\@: jaune
\item Grand Oral: rose
\item Soutenance: violet
\end{itemize}


%image


Ce tableau de bord du projet est accessible et visible en permanence de l’ensemble de l'équipe projet.
Il va permettre de suivre en temps réel l'évolution des tâches et des user stories à réaliser.
Le scrum board est séparé au minimum en quatre parties: Le product backlog, les tâches à faire, les tâches en cours et les tâches terminées.

Nous avons ajouté une phase `En cours de validation' afin de vérifier que les tâches ont bien été faite.


\subsubsection{scrum}
La méthode agile permet de délivrer un projet/produit très rapidement.
Ce cadre méthodologique est conçu sur des cycles de développement court durant lesquels on s'adapte constamment tout maintenant l'utilisateur au centre. 

Elle se compose de:
\begin{itemize}
\item Scrum master
\item Product Owner
\item développeurs
\end{itemize}


Le scrum master est garant du processus scrum.
Il s'assure d’une bonne communication entre les membres de l'équipe.
Afin que l'équipe comprenne bien cette méthodologie de travaille, le scrum master a écrit un document qui l'explique en détail.

De plus elle se compose aussi d'un product owner qui représente le client.
Il définit les spécifications fonctionnels (exemple:  Integration Validation Verification Qualification) et établie la liste des priorités de ce qu'il faut développer.
C'est aussi lui qui valide les fonctionnalités. 

\paragraph{User Story}
Le processus Scrum commence par la réalisation d'une User Story.
Elle décrit l’expérience utilisateur en utilisant le langage, le vocabulaire et la terminologie de l’usager. 

Une User Story comporte:
\begin{itemize}
\item Un identifiant: un nom contenant la fonction du produit de manière succincte
\item L'importance: Une valeur qui définit la priorité de la story
\item Estimation du travail nécessaire
\item Démonstration: Un test simple de la story qui sera à valider
\end{itemize}


Nous avons décrit l’ensemble des exigences du commanditaire en user story de la manière suivante:


%image		%image


Cette étape nous a permis de nous recontextualiser le PoC demandé par le commanditaire.

\paragraph{Product Backlog}
De l'User Story va émaner des exigences.
Elles seront hiérarchisées avec le client dans un product backlog.
Nous pouvons le voir comme un carnet de commande pour le produit.
C’est un miroir de ce qu'il faut faire pour réaliser les besoins du client et délivrer l'User Story.

Le product backlog va constamment évoluer pour refléter les nouveaux besoins.  
En listant l'ensemble des tâches dans un document word, nous avons pu catégoriser l’ensemble des tâches. 


%image


\paragraph{Sprint}
Une fois d'accord avec l'User Story et les exigences (product backlog), il est temps de se lancer dans la réalisation du projet.
Il sera découpé en plusieurs itérations que l’on nomme des sprints.  

Les étapes du sprint sont: 
\begin{itemize}
\item \textbf{Sprint planning meeting}

Un sprint commence par une réunion de planification.
Au cours de cette séance, on va aller puiser les éléments prioritaires du product backlog qui seront développés dans les sprints.
Aussi nommé daily meeting, nous avons utilisé cette méthode.
Nous avions réunions du tous les matins à partir de 10h (lundi – vendredi) afin d’attribuer les tâches de la journée.
Nous avons rajouté une autre réunion le soir à partir de 19h qui n’est pas prévu dans la méthode Scrum.
Cette dernière réunion nous a permis de revoir les difficultés que nous avons rencontré et de discuter des tâches de la journée.

\item \textbf{Sprint Backlog}

Dans chaque sprint qui durent entre 1 et 2 semaines, il y aura du développement puis un contrôle qualité (du test) et une livraison.
L'ensemble des livraisons des sprints cumulés se nomme le sprint backlog.
C'est la réunion avec le commanditaire par équivalence.
Cela nous a permis de proposer deux versions de PoC

\end{itemize}

\subsubsection{Horaires}
Nos horaires de travails étaient en moyenne de 10h-18h avec une pause déjeuner vers 12h.

Nous avions une réunion tous les matins à partir de 10h et un le soir à partir de 19h.


\subsection{Charge de travail}
\textbf{Prévisionnel}


En listant l'ensemble de tâches, nous avons établi leurs charges de travail en jours/hommes et en heures. 

Nous avons définis les tâches suivantes:
\begin{itemize}
\item Sprint 1 (06/01/2020 - 10/01/2020):
\begin{itemize}
\item Métadonnées: 6 jours/hommes pour 2 personnes
\item Taxonomie: 18 jours/hommes pour 1 personnes
\item Moteur de recherche: 15 jours/hommes 2 personnes 
\end{itemize}

\item Sprint 2 (13/01/2020 - 03/02/2020):
\begin{itemize}
\item DER\@: 11 jours/hommes pour 3 personnes
\item Grand Oral: 6 jours/hommes pour 3 personnes
\item Soutenance:  9 jours/hommes pour 3 personnes 
\end{itemize}
\end{itemize}


Total:  Nous avons prévu un total de 65 jours/hommes de travail en 3 semaines pour 3 personnes. 


\textbf{Réelle}
\begin{itemize}
\item Sprint 1 (06/01/2020 - 10/01/2020):
\begin{itemize}
\item Métadonnées: 9 jours/hommes pour 2 personnes
\item Taxonomie: 12 jours/hommes pour 1 personnes
\item Moteur de recherche: 31 jours/hommes pour 2 personnes 
\end{itemize}

\item Sprint 2 (13/01/2020 - 05/02/2020):
\begin{itemize}
\item DER\@: 18 jours/hommes pour 3 personnes
\item Grand Oral: 8 jours/hommes pour 3 personnes
\item Soutenance:  9 jours/hommes pour 3 personnes
\end{itemize}
\end{itemize}

Total:  Nous avons effectué un total de 127 jours/hommes en 3 semaines pour 3 personnes. 
Ci-dessous nous pouvons voir la comparaison des charges de travail réel contre ce dont nous avons prévu en jours/hommes: 


%image


\subsection{Gantt prévisionnel/réel}
Au début du projet XL, nous avons effectué un diagramme de Gantt prévisionnel afin de comparer les charges de travails ainsi que la durée de chacune des taches.
Le diagramme du projet XL commence le 06/01/2020 et se termine le 05/02/2020.

Sur l'image ci-dessous, nous avons listé certaines taches datant du 02/12/2019 nous avons quelques dépendances pour les taches du début du projet XL\@.
De plus, on peut remarquer que nous avons des indications de pourcentage pour le projet et les taches.
Dans notre Trello, lorsqu'une tâche est dans le product backlog ou dans la section `To do' elle est automatiquement à 0\%.
En passant dans `In progress' elle passe à 20\%, dans `En cours de validation' à 80\% puis à 100\% lorsqu'elle est dans `Done'.
Ainsi nous avions un diagramme dynamique avec une métrique plus intéressante comparé à notre ancien Gantt. 



Ci-dessous nous avons le diagramme de Gantt prévisionnel et réel: 


%image


%image


\subsubsection{Sprint 1: Durée Réel/Prévisionnel}
A l'aide de Trello, nous avions établi un calendrier hebdomadaire.
Ce dernier nous a permis d'avoir une indication visuelle sur le nombre de tâches que nous avons par jours.
Si nous devions comparer les deux images ci-dessous (prévisionnel et réel), nous pouvons remarquer que certaines tâches ont pris moins de temps que prévues.
Cependant, la quantité de travails par jours est plus importantes.


%image


%image


\subsubsection{Sprint 2: Durée Réel/Prévisionnel}
Le deuxième sprint est plutôt consacré au DER, au Grand Oral ainsi qu'à la soutenance.
Nous n'avons pas prévu que le DER aurait pris autant de temps (visible sur les images ci-dessous).
Comme résultat, nous nous sommes retrouvés a effectuer plus de tâches que prévu par jours.


%image


%image


Même si le diagramme de Gantt n'a pas été respecté tel que prévu, le fait d’avoir estimé les charges de travails ainsi que d'établir un calendrier de travail cela nous a permis d'être à temps sur l’ensemble du projet.
Nous pouvons en déduire que la méthode agile a accéléré l'avancement du projet.
