%Description des fonctions à servir (diagramme des cas d'utilisation)


\subsection{Recherche d'informations d'importance} %Baptiste
Chaque document contient des informations d'importance qui doivent être extraites pour que le document en question soit correctement classé par la suite.
Les informations d'importance, déterminées avec le commanditaire, sont les suivante :
\begin {itemize}
\item La référence du RAA
\item La date de publication
\item Les dates
\item Les références des arrêtés
\item Les références des décrets
\item Les références des articles
\item Les références des lois
\item Les noms des parties prenantes
\item Les lieux
\end {itemize}

\subsection{Classement par taxonomie}
Pour pouvoir obtenir une classification précise des documents administratifs, une taxonomie contenant plus de 6000 termes a été développée par l'administration Française. Cette taxonomie permet de tagger précisément le contenu d'un document.
Il était donc impératif de pouvoir extraire depuis le texte assez d'information pour pouvoir assigner au document une ou plusieurs taxonomies.

\subsection{Moteur de recherche}

