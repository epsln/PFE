
\section{Avant-Projet XL}

\subsection{Objectifs SMART}
Afin de mieux délimiter le projet et ses objectifs, nous avons convenu avec le client d'un certain nombre d'objectifs, correspondants au critères SMART\@.
Ces objectifs de projet nous ont permis de mieux juger de la charge de travail nécessaire et nous ont donnés des buts concrets.
Nous étions donc également en mesure de pouvoir quantifier l'avancement du projet.

Ces objectifs sont:
\begin{itemize}
\item Construire un système d'extraction du texte brut depuis des PDF
\item Extraire une liste de métadonnées (définie par le client) depuis le texte brut
\item Assigner au moins une taxonomie au texte
\item Indexer les documents de notre base de données à partir des informations extraites (métadonnées et taxonomies)
\item Interfaçage avec un moteur de recherche
\end{itemize}


Nous étions capables de vérifier si chacun de ces objectifs était accomplis grâce à une série de test, définie section 8 du DER\@.


\subsection{Gantt}
Dans un premier temps, nous avions effectué un planning de Gantt post mortem avec les grandes tâches.
Cette étape a été l'une de nos premières tâches.
Cette organisation permet d’avoir une estimation de la charge de travail.

Avec la complexité du projet, nous voulions nous consacrer sur les tâches les plus importantes.
Cependant, il valait mieux avoir un diagramme général afin d’avoir un support visuel pour évaluer la charge de travail restante.


%image


Le schéma ci-dessus est le diagramme de Gantt qui nous a permis de nous aiguiller tout le long du projet.
En orange nous avons la durée des tâches que nous avons estimées et en gris leurs durées réelles. 

Nous pouvons remarquer que nous sommes généralement dans les temps même si certaines tâches ont pris plus de temps que prévues.
L'analyse sémantique représente la taxonomie qui est la charge de travail la plus importantes.  

L'analyse visuel nécessaire a été supprimé en cours de route car suite à une réunion avec le commanditaire, il nous a été demandé de centrer le POC sur un seul type de document.
De ce fait, nous avons gagné plus de temps pour nous focaliser sur d'autres tâches.  

\subsection{Gantts Hebdomadaire}
Le diagramme représenté dans la partie a) est très bien pour avoir une vue globale sur l'avancement du projet.
Cependant, pour chaque grande tâche dépendent des tâches plus petites.  

C'est la raison pour laquelle nous avons fait un diagramme hebdomadaire divisé en sept sprints.


%image


Cette démarche nous a été d'une grande aide afin de nous guider sur notre avancement.
Ci-dessus, nous avons l’exemple du sprint 3/7 qui concerne en partie de notre demande de projet XL et l'implémentation d’une OCR\@.

Nous avons des indications sur la durée que nous avons prévu (en orange) et le retard que nous avons pris (en gris).  

\subsection{Organisation}
Deux membres de l'équipes travaillaient depuis Laval et l'autre depuis Paris.
Cela ne nous a pas dérangé, cette situation nous a permis d'organiser des réunions régulières avec le commanditaire.

\subsubsection{SharePoint et GitHub}
Afin de nous organiser, l'ensemble des fichiers non-techniques étaient stockés sur le Sharepoint Cap Projet.
Ce dernier se compose des documents administratifs, de nos comptes rendu de réunions, des recueils administratifs de la préfecture et d'un trombinoscope.

Il nous permet de nous structurer et d'informer notre mentor lorsqu'un document est disponible. 


En ce qui concerne de nos programmes, nous avons fait le choix de travailler avec Github qui est un service web d'hébergement et de gestion de développement de logiciels.
Ainsi, nous avions un contrôle de version des codes développés.

\subsubsection{Réunions}
Pour un suivi régulier du projet, nous faisions une réunion par semaine avec notre mentor.
Avec la distance de travail entre les membres de l'équipe et le mentor (Paris-Laval), nous communiquions en visioconférence.

C'est avec des PowerPoints que nous présentions notre avancement.
Cette démarche a permis d’avoir une réunion structurée, lisible et présentable. 


%image


Au début du projet, nous avions accès à un fichier Excel sur le SharePoint.
Dans un premier temps, nous avions évalué les charges de travail qui seront nécessaire pour chaque tâche.
Cette démarche nous a permis de réaliser l'ampleur de ce projet et par la suite de faire une demande de projet XL\@.  

Nous avons établi la charge prévisionnelle suivante :  


%image

Nous remarquons que nous avons pris de la marge sur l'évaluation des tâches en jours/hommes à cause des différents risques que nous pourrions rencontrer.
La charge de travail réel est deux fois moins que ce que nous avons évalué.  

En réalité, nous avons effectué une charge de travail équivalent à 504 heures.








