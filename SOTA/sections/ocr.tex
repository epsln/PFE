\subsection{\href{https://github.com/tesseract-ocr/tesseract}{Tesseract}}
Tesseract est une suite de software complète, comprenant un moteur d'OCR: libtesseract, ainsi qu'une interface en CLI. Tesseract est un des plus vieux projets de reconnaissance optique de charactère; il est en developpement depuis 1985, initialement par HP Labs. Le projet est désormais open source et est developpé par google. 

La dernière version de Tesseract, Tesseract 4, est basé sur un réseau neuronal récurrent, de type Long Short Term Memory (LSTM), qui se concentre sur la reconnaissance des lignes. 

Tesseract 4 supporte l'unicode (UTF-8) et peut reconnaitre plus de 100 langages. De plus, Tesseract peut avoir une sortie dans une grande variété de format de fichier, comme du texte simple, du hOCR, pdf... 

Tesseract constitue l'état de l'art en matière de précision et rapidité.

\subsection{\href{https://github.com/tmbdev/ocropy}{Ocropy}}
Ocropy est une collection de programme d'analyse de documents. On y trouve des programmes aidant au preprocessing, comme pour la binarisation et de la reconnaissance de la mise en page du document. On y trouve également de nombreux modèles, pouvant être adaptés a des polices d'écriture différentes. Le moteur d'OCR est programmé pour être parallélisable.

\subsection{\href{https://www.gnu.org/software/ocrad/}{Ocrad}}
Ocrad est le programme d'OCR de GNU, qui se base sur une méthode d'extraction des features. Il permet de lire des images en bitmap et produit une sortie en texte UTF-8.
